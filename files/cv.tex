% !TeX spellcheck = en_US
\documentclass[11pt,a4paper]{moderncv}
\moderncvtheme[blue]{classic}
\usepackage[utf8]{inputenc}
\usepackage[top=1.5cm, bottom=1.5cm, left=1.5cm, right=1.5cm]{geometry}


% Largeur de la colonne pour les dates
\setlength{\hintscolumnwidth}{2,3cm}

% Une entête classique
\firstname{David}
\familyname{Guyon}
\title{Ph.D Graduate in Computer Science}
%\quote{Feel free to contact me at david@guyon.me}
\address{Lieu-dit Logne}{44 240, Sucé sur Erdre}
\email{david.guyon@inria.fr}

\begin{document}

\makecvtitle

\section{Educations}
\cventry{2015 -- 2018}
  {Ph.D in Computer Science}{ED MathSTIC, Université de Rennes 1}{Rennes}{}
	{Supporting Energy-awareness for Cloud Users\\Under the supervision of Christine Morin and Anne-Cécile Orgerie}
\cventry{2014 -- 2015}
  {Second year Master’s degree in Computer Science Research}{ISTIC, Université de Rennes 1}{Rennes}{}
  {French mention \textit{Assez bien}, thesis topic presented below}
\cventry{2013 -- 2014}
  {First year Master’s degree in Software Engineering}{ISTIC, Université de Rennes 1}{Rennes}{}
  {French mention \textit{Assez bien}, $11^{th}$ in the ranking}
\cventry{2012 -- 2013}
  {BEng in Computer Science}{Edinburgh Napier University}{Scotland}{}
  {Top 10\% of the promotion}
\cventry{2010 -- 2012}
  {University Diploma in Technological Sciences}{Angers/Cholet University}{}{}
	{Electronic engineering and industrial computing, $2^{nd}$ in the ranking}
\cventry{2010}
  {Baccalauréat STI Génie Électrotechnique}{NDBN High School}{Beaupréau (49)}{}{}
%\cventry{2006 -- 2008}
%  {BEP Électrotechnique}{Lycée technique Le Pinier Neuf}{Beaupréau}{}{}

%\section{Formations}

%\subsection{General Formations}
%\cventry{Nov. 2016}
%  {Panorama d'outils et démarche pour développer une veille personelle}{URFIST, Université de Rennes 2}{6 hours}{}{}
%\cventry{Sept. 2016}
%  {Gagner en assurance et détente lors de vos soutenances, conférences, réunions...}{Inria Rennes Bretagne Atlantique}{14 hours}{}{}

%\subsection{Scientific Formations}
%\cventry{October 2018}
%  {E3-RSD 2018}{Dinard}{15 hours}{}{}
%\cventry{June 2016}
%  {Histoire des Sciences}{IRMAR, Université de Rennes 1}{4,5 hours}{}{}
%\cventry{May 2016}
%  {E3-RSD 2016}{Dinard}{20 hours}{}{}
%\cventry{Feb. 2016}
%  {Winter School Grid'5000 2016}{Inria Grenoble}{12 hours}{}{}
%\cventry{Nov. 2015}
%  {Internet des Services (ISI)}{EIT ICT Digital}{16 hours}{}{}

\section{Professional Experiences}
\cventry{April 2019\\present}
  {Post-doctoral researcher}
  {IMT-Atlantique, Inria, LS2N}{Nantes}{}
  {Working with Shadi Ibrahim on topics related to Data Stream Processing (DSP) systems. Our goals are a better understanding on the origin of slow down of DSP applications (i.e. \textit{stragglers}), and an improved handling in correcting such kind of performance loss.}
\cventry{Sept. 2015\\Nov. 2018}
  {Ph.D in the Myriads team}
  {Université de Rennes 1, Inria, CNRS, IRISA}{Rennes}{}
  {My thesis is titled \textit{Supporting Energy-awareness for Cloud Users}. It was financed for 3 years by the Université de Rennes 1, and then by Inria during a 3 months extension (``\textit{relais thèse}'' from September 2018 to November included).\\\newline
  My thesis focuses on the inclusion of users in systems that deal with the energy consumption of cloud computing datacenters. The idea is that an informed user about her environmental impact can be led to undertake ecological actions. The approach considered in this thesis was, initially, to model an energy information that goes back to users to motivate ecological actions. Then, in a second step, to provide users with means of action to reduce their energy impact at the cost of a loss of performance (e.g. to delay the obtaining of execution results). Contributions from this thesis, located at different levels of the cloud services stack (IaaS and PaaS) have shown that user participation through energy/performance tradeoffs reduces the energy consumption of cloud computing systems.}

\cventry{}{}{}{}{}
  {\vspace{-2em}Jury members composition: 
  \begin{itemize}
    \item Jean-Marc Menaud, professor, IMT-Atlantique, LS2N, \textbf{president}
    \item Sébastien Monnet, professor, Université Savoie Mont Blanc, LISTIC, \textbf{reviewer}
    \item Jean-Marc Pierson, professor, Université de Toulouse, IRIT, \textbf{reviewer}
    \item Laurent Lefèvre, chargé de recherche, Université de Lyon, LIP, \textbf{examiner}
    \item Guillaume Pierre, professor, Université de Rennes 1, IRISA, \textbf{examiner}
    \item Sébastien Varrette, chargé de recherche, Université du Luxembourg, \textbf{examiner}
    \item Christine Morin, directrice de recherche, Inria, IRISA, \textbf{thesis director}
    \item Anne-Cécile Orgerie, chargée de recherche, CNRS, IRISA, \textbf{thesis co-director}
  \end{itemize}}
\vspace{0.6em}

\cventry{Feb. 2015\\June 2015}
  {Master 2 internship in the Myriads team}
  {Inria, IRISA}{Rennes}{}
  {Supervised by Christine Morin and Anne-Cécile Orgerie, my internship subject was to design a cloud infrastructure system (IaaS) that sees the user as a lever to reduce the electrical energy consumption of datacenters. The user selects an execution mode for her application that will allow to use more or less resources on the cloud. Using fewer resources makes it easier to consolidate on fewer servers. Experimental results show that this approach can reduce power consumption (because fewer servers are needed) for slightly longer user application execution time.}
\vspace{0.6em}

\cventry{May 2014\\July 2014}
  {Master 1 internship in the ALF team}
  {Inria, IRISA}{Rennes}{}
  {As part of the ALF team (now PACAP) and under the supervision of Erven Rohou, I developed an extension of \textit{hwloc}, a tool that provides a graphical interface of the processor architecture on which it is running. The extension named \textit{dynamic lstopo} displays caches and CPUs current usage level using performance hardware counters to help understanding the behavior of the processor. This work was the subject of an article presented at the ICCS 2015 conference.}
\vspace{0.6em}

\cventry{Sept. 2013\\May 2014}
  {CROWD project}
  {ISTIC}{Rennes}{}
  {With a team of 11 students and under the direction of David Gross-Amblard (IRISA, DRUID team), we built a workflow engine for a crowdsourcing web application. As an illustration, in case of emergency we can ask the crowd to indicate on a map the blocked roads and then give this map as input of the following workflow elements. The map will allow to find volunteers nearby the blocked roads to help the authorities. Following the completion of the project, I participated in the writing of an article presented at the 2014 BDA conference.}
\vspace{0.6em}
  
\cventry{Sept. 2012\\May 2013}
  {Third year group project}{Edinburgh Napier University}{Scotland}{}
  {With a team of 5 students, we developed a web site and an Android application to provide to the students of our University to book for a taxis in Edinburgh.}
\vspace{0.6em}
  
%\cventry{April 2012\\July 2012}
%  {Stage universitaire en R\&D}
%  {Company NGV Électronique}{Cholet (49)}{}
%  {Developer in the R\&D department for a new BusCAN module (Hardware and Software). I searched for components, drew the schematic and studied an existing C kernel for a PIC microcontroller. Then I wrote the source code for a kernel update. Internship subject is confidential.}
%\vspace{1.2em}

\section{Teaching Missions}

\cventry{Feb. 2020}{Green IT}{IMT-Atlantique}{Nantes}{}
  {Lectures and practical sessions for the A3 GSI students to explain the energy flow in datacenters and its possible optimizations. We also cover the life cycle of datacenters to show their environmental and societal impacts. This course was co-organized with Jean-Marc Menaud.}
\vspace{0.6em}

\cventry{June 2019}
  {FIL A1 Group Project}{IMT-Atlantique}{Nantes}{}
  {Project supervision for the FIL A1 students. A group of 4 students had to implement a game where the players have to guess a word selected by the computer. Each player can participate through the network. This experience included project supervision and final evaluation.}
\vspace{0.6em}

\cventry{$1^{st}$ semester 2015/2016}
  {Coopération et concurrence dans les systèmes et réseaux (CSR)}{ISTIC}{Rennes}{}
  {Over 2 consecutive years, I provided teaching hours TD/TP for the CSR course supervised by Cédric Tedeschi for the students in Master 1 MIAGE.}
\vspace{0.6em}

\cventry{$2^{nd}$ semester 2015/2016}
  {Organisation et utilisation des systèmes d'exploitation 2 (SYR2)}{ISTIC}{Rennes}{}
  {Over 2 consecutive years, I provided teaching hours TD/TP for the SYR2 course supervised by Guillaume Pierre for the students in Licence 3 Info.}

\newpage

\section{Publications}

The following articles have been published in peer-reviewed scientific conferences with proceedings. You can click on the conference acronyms in order to be redirected to the corresponding website. 

\subsection{International Conferences}

\cvitem{Oct. 2020}{
  {Thomas Lambert, David Guyon and Shadi Ibrahim},
  {\textbf{Rethinking Operators Placement of Stream Data Application in the Edge}},
  \textit{ACM International Conference on Information and Knowledge Management}
  (\href{https://www.cikm2020.org/}{CIKM 2020}), \textit{short paper}
}
\vspace{0.6em}

\cvitem{Sept. 2018}{
  {David Guyon, Anne-Cécile Orgerie and Christine Morin},
  {\textbf{Energy-efficient IaaS-PaaS Co-design for Flexible Cloud Deployment of Scientific Applications}},
  \textit{International Symposium on Computer Architecture and High Performance Computing} (\href{https://graal.ens-lyon.fr/sbac-pad/}{SBAC-PAD 2018}), rank B in CORE2018 (26\%)
}
\vspace{0.6em}

\cvitem{April 2018}{
	{David Guyon, Anne-Cécile Orgerie and Christine Morin},
	{\textbf{An Experimental Analysis of PaaS Users Parameters on Applications Energy Consumption}},
	\textit{IEEE International Conference on Cloud Engineering} (\href{http://conferences.computer.org/IC2E/2018/}{IC2E 2018}), \textit{short paper}
}
\vspace{0.6em}

\cvitem{April 2017}{
	{David Margery, David Guyon, Anne-Cécile Orgerie, Christine Morin, Gareth Francis, Charaka Palansuriya and Kostas Kavoussanakis},
	{\textbf{A CO2 emissions accounting framework with market-based incentives for Cloud infrastructures}},
	\textit{International Conference on Smart Cities and Green ICT Systems} (\href{http://www.smartgreens.org/?y=2017}{SMARTGREENS 2017}), \textit{short paper}
}
\vspace{0.6em}

\cvitem{March 2017}{
	{David Guyon, Anne-Cécile Orgerie, Christine Morin and Deb Agarwal},
	{\textbf{How Much Energy can Green HPC Cloud Users Save?}},
	\textit{Euromicro International Conference on Parallel, Distributed, and Network-Based Processing} (\href{https://www.pdp2017.org/}{PDP 2017}), rank C in CORE2018 (49\%), \textit{short paper}
}
\vspace{0.6em}

\cvitem{Dec. 2015}{
	{David Guyon, Anne-Cécile Orgerie and Christine Morin},
	{\textbf{Energy-efficient User-oriented Cloud Elasticity for Data-driven Applications}},
	\textit{IEEE International Conference on Green Computing and Communications} (\href{http://www.swinflow.org/confs/greencom2015/}{GreenCom 2015})
}
\vspace{0.6em}

\cvitem{June 2015}{
	{Erven Rohou and David Guyon},
	{\textbf{Sequential Performance: Raising Awareness of the Gory Details}},
	\textit{International Conference on Computational Science} (\href{https://www.iccs-meeting.org/iccs2015/}{ICCS 2015}), rank A in CORE2018 (14\%)
}

\subsection{Journal}

\cvitem{Feb. 2018}{
  {David Guyon, Anne-Cécile Orgerie, Christine Morin and Deb Agarwal},
  {\textbf{Involving Users in Energy Conservation: A Case Study in Scientific Clouds}},
  \textit{\href{http://www.inderscience.com/jhome.php?jcode=ijguc}{International Journal of Grid and Utility Computing (Inderscience)}}, 14 pages
}

\subsection{Posters and Workshops}

\cvitem{April 2018}{
  {David Guyon, Anne-Cécile Orgerie and Christine Morin},
  {poster session at the \href{http://conferences.computer.org/IC2E/2018/}{IC2E 2018} conference presented by Anne-Cécile Orgerie (\href{https://david.guyon.me/files/poster-ic2e.pdf}{poster available here})}
}
\vspace{0.6em}

\cvitem{Nov. 2017}{
  {David Guyon},
  {poster session on my thesis work at the D1 departement (Large-Scale Systems) day in which the Myriads team is attached to (\href{https://david.guyon.me/files/poster-D1.pdf}{poster available here})}
}
\vspace{0.6em}

\cvitem{May 2017}{
  {David Guyon, Anne-Cécile Orgerie and Christine Morin},
  {\textbf{GLENDA: Green Label towards Energy proportioNality for IaaS DAta centers}},
  \textit{International Workshop on Energy-Efficient Data Centres} (\href{https://e2dc.eu/events.html\#event2017}{E2DC 2017})
}

\subsection{National Conference}

\cvitem{Oct. 2014}{
  {Ahmad Chettih, David Gross-Amblard, David Guyon, Erwann Legeay and Zolt\'{a}n Mikl\'{o}s},
  {\textbf{Crowd, a platform for the crowdsourcing of complex tasks}},
  \textit{Base de Données Avancées} (\href{http://bda2014.imag.fr/}{BDA 2014})
}

%\section{Compétences}
%\cvitem{Langages}
%  {C/C++, Java, Scala, OCaml, Python/Django, HTML5, CSS3, JS, SQL, PHP}
%\cvitem{Systèmes}
%  {Linux (Debian, Archlinux), Windows (toutes versions)}
%\cvitem{Logiciels/outils}
%  {Eclipse, Vim, Emacs, Sublime Text, \LaTeX{}, PyCharm, Qt Creator}
%\cvlanguage{Anglais}
%  {\textnormal{lu, écrit, parlé} -- CLES de niveau 2 (2014)}{}

%\section{Centres d'intérêt}
%
%\cvitem{Général}
%{Sur mon temps libre j'apprécie faire de la guitare, du piano et du chant. En dehors du monde 
%musical, je me lance souvent dans des petits projets de bricolage ou de programmation.}
%
%\cvitem{Associatif}
%  {Bénévole dans l'association Animaje qui participe à l'organisation du festival 
%  \textit{Hellfest}. Secrétaire et Trésorier de l'association \textit{Feu de Camp} dont le but est 
%  d'organiser un événement annuel pour faire connaitre de jeunes groupes de musique locaux.}

\end{document}
